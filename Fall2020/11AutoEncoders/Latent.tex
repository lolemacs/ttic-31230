\input ../SlidePreamble
\input ../preamble


\begin{document}

{\Huge

  \centerline{\bf TTIC 31230, Fundamentals of Deep Learning}
  \bigskip
  \centerline{David McAllester, Winter 2020}
  \vfill
  \vfil
  \centerline{\bf Interpretable Latent Variables}
  \vfill
  \vfill

\slide{Latent Variables}

{\color{red} $$P_\Phi(y) = \sum_z\;P_\Phi(z)P_\Phi(y|z) = E_{z \sim P_\Phi(z)}\;P_\Phi(y|z)$$}

Or

\vfill
{\color{red} $$P_\Phi(y|x) = \sum_z\;P_\Phi(z|x)P_\Phi(y|z,x) = E_{z \sim P_\Phi(z|x)}\;P_\Phi(y|z,x)$$}

\vfill
Here {\color{red} $z$} is a latent variable.

\slide{Interpretable Latent Variables}

{\color{red} $$P_\Phi(y) = \sum_z\;P_\Phi(z)P_\Phi(y|z) = E_{z \sim P_\Phi(z)}\;P_\Phi(y|z)$$}

Here we often think of $z$ as the causal source of $y$.

\vfill
For example $z$ might be a physical scene causing image $y$.

\vfill
Or $z$ might be the intended utterance causing speech signal $y$.

\vfill
In these situations a latent variable model should more accurately represent the distribution on $y$.

\slide{Interpretable Latent Variables}

{\color{red} $$P_\Phi(y) = \sum_z\;P_\Phi(z)P_\Phi(y|z) = E_{z \sim P_\Phi(z)}\;P_\Phi(y|z)$$}

\vfill
$P_\Phi(z)$ is called the prior.

\vfill
Given an observation of $y$ (the evidence) $P_\Phi(z|y)$ is called the posterior.

\vfill
Variational Bayesian inference involves approximating the posterior.

\anaslide{Colorization with Latent Segmentation}
\medskip
\centerline{\includegraphics[width = 5in]{../images/colorizationGreg2}}
\centerline{$x$ \hspace{4em} $\hat{y}$ \hspace{4em} $y$}
\centerline{\huge Larsson et al., 2016}

\vfill
Colorization is a natural self-supervised learning problem --- we delete the color and then try to recover it from the grey-level image.

\vfill
Can colorization be used to learn segmentation?

\vfill
Segmentation is latent --- not determined by the color label.

\anaslide{Colorization with Latent Segmentation}
\medskip
\centerline{\includegraphics[width = 5in]{../images/colorizationGreg2}}
\centerline{$x$ \hspace{4em} $\hat{y}$ \hspace{4em} $y$}
\centerline{\huge Larsson et al., 2016}

\vfill
$x$ is a grey level image.

\vfill
$y$ is a color image drawn from $\pop(y|x)$.

\vfill
$\hat{y}$ is an arbitrary color image.

\vfill
$P_\Phi(\hat{y}|x)$ is the probability that model $\Phi$ assigns to the color image $\hat{y}$ given grey level image $x$.

\anaslide{Colorization with Latent Segmentation}
\medskip
\centerline{\includegraphics[width = 5in]{../images/colorizationGreg2}}
\centerline{$x$ \hspace{4em} $\hat{y}$ \hspace{4em} $y$}

\vfill
{\color{red} $$P_\Phi(\hat{y}|x) = \sum_z\;P_\Phi(z|x)P_\Phi(\hat{y}|z,x).$$}
\begin{eqnarray*}
\mbox{input}\; x \\
P_\Phi(z|x) & = & \ldots \;\;\mbox{\color{red} semantic segmentation} \\
P_\Phi(\hat{y}|z,x) & = & \ldots \;\;\mbox{\color{red} segment colorization} \\
\end{eqnarray*}

\anaslide{Assumptions}

\bigskip
\bigskip
We assume models $P_\Phi(z)$ and $P_\Phi(y|z)$ are both samplable and computable.

\vfill
In other words, we can sample from these distributions and for any given $z$ and $y$ we can compute $P_\Phi(z)$ and $P_\Phi(y|z)$.

\vfill
These are nontrivial assumptions.

\vfill
A loopy graphical model is neither (efficiently) samplable nor computable.

\slide{Cases Where the Assumptions Hold}

In CTC we have that $z$ is the sequence with blanks and $y$ is the result of removing the blanks from $z$.

\vfill
In a hidden markov model $z$ is the sequence of hidden states and $y$ is the sequence of emissions.

\vfill
An autoregressive model, such as an autoregressive language model, is both samplable and computable.

\anaslide{Image Generators}

\bigskip
\centerline{$z$\hspace{5in}$y_\Phi(z)$}
\centerline{\includegraphics[width=6in]{\images/generator2}}

\bigskip
We can generate an image $y$ form noise $z$ where $p_\Phi(z)$ and $p_\Phi(y|z)$ are both samplable and computable.

\bigskip
Typically $p_\Phi(z)$ is ${\cal N}(0,I)$ reshaped as $z[X,Y,J]$


\anaslide{Image Generators}

\bigskip
\centerline{$z$\hspace{5in}$y_\Phi(z)$}
\centerline{\includegraphics[width=6in]{\images/generator2}}

\bigskip
Our assumptions hold for image generators such as GANs, but $z$ is typically viewed as ``noise'' and is not interpretable.

\anaslide{Modeling $y$}

\vfill
We would like to use the fundamental equation

$$\Phi^* = \argmin_\Phi\;E_{y \sim \pop}\;- \ln P_\Phi(y)$$

\bigskip
\bigskip
But even when $P_\Phi(z)$ and $P_\Phi(y|z)$ are samplable and computable we cannot typically compute $P_\Phi(y)$.

\vfill
Specifically, for $P_\Phi(y)$ defined by a generator we cannot compute $P_\Phi(y)$ for a test image $y$.
\slide{END}

\end{document}
