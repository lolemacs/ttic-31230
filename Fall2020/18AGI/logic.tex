\input ../SlidePreamble
\input ../preamble

\begin{document}

{\Huge

  \centerline{\bf TTIC 31230, Fundamentals of Deep Learning}
  \bigskip
  \centerline{David McAllester, Winter 2020}

\vfill
  \centerline{\bf Representing Functions and Knowledge with Logic}
  \vfill
  \vfill

\slide{The Procedural vs. Declarative Debate}

Any Discussion today of the ``knowledge representation problem'' is likely to entail a debate between proponents of {\bf declarative} and {\bf procedural} representations of knowledge.

\vfill
\rightline{Terry Winograd, 1974}

\vfill
Programming Languages (Procedural)

\vfill
vs. Logic or Natural Language (Declarative)

\slide{Is Human Common Sense Based on Logic?}

Certain facts are obvious.

\vfill
A king on empty chess board can reach every square (obvious).

\vfill
A knight on an empty chess board can reach every square (true but not obvious).

\slide{Obviousness}

Consider a graph with colored nodes.

\vfill
If every edge is between nodes of the same color, then any path connects nodes of the same color.

\vfill
Consider a swiss chocolate bar of $3 \times 5$ little squares.

\vfill
How many breaks does it take to reduce this to fifteen unconnected squares?

\vfill
What inference happens when one observes that each break increases the number of pieces by one?

\slide{Logical Representations of Events}

Let $e$ range over ``events''.

\vfill
$e:\mathrm{give}(a_1,x,a_2) \Rightarrow \mathrm{had}(a_1,x,\mathrm{before}(e)) \;\wedge\; \mathrm{had}(a_2,x,\mathrm{after}(e))$

\vfill
This is related to Davidsonian semantics for natural language (1969) and the situation calculus of McCarthy and Hayes (1968).

\slide{Bottom-up Logic Programming}

Bottom-up logic programming is distinguished by its relationship to dynamic programming algorithms.

\vfill
$$\mathrm{At}(x) \Rightarrow \mathrm{Reachable}(x)$$

\vfill
$$\mathrm{Reachable}(x) \wedge \mathrm{CanGo}(x,y) \Rightarrow \mathrm{Reachable}(y)$$

\vfill
This defines a linear time algorithm for reachability.

\slide{Datalog}

A set of inference rules each of which has antecedents and conclusions that are just predicates applied to variables is called a {\bf datalog} program.

\vfill
It can be shown that datalog ``captures the complexity class $P$'' --- they can express {\bf all and only} polynomial time decidable relations
(provided the entities are assigned a total order).

\vfill
General bottom-up logic programs, including expressions (terms), are Turing complete.

$$N(x) \Rightarrow N(s(x))$$

\slide{Type Theoretic Foundations of Mathematics}

  $$
  \begin{array}{|l|c|c|c|c|}
  \hline
    \mbox{variables, pairs} & x & (e_1,e_2) & \pi_i(e) \\ \hline 
    \mbox{functions} & \lambda \intype{x}{\sigma}\;e[x] & f(e) & \\ \hline 
    \mbox{atoms} & ~~P(e)~~~ & ~~ e_1 \doteq e_2 ~~~~ & e_1 =_\sigma e_2 \\ \hline
    \mbox{formulas} & \neg \Phi &  \Phi_1 \vee \Phi_2 & \forall \intype{x}{\sigma}\; \Phi[x] \\ \hline
    \mbox{types}   & \Sigma_{\intype{x\;}{\;\sigma}}\;\tau[x] & \Pi_{\intype{x\;}{\;\sigma}}\;\tau[x] & S_{\intype{x\;}{\;\sigma}}\;\Phi[x]  \\ \hline
    \mbox{type constants} & \mathrm{Bool} & \mathrm{Set} & \mathrm{Type} \\ \hline
  \end{array}
  $$



\slide{The Substitution of Isomorphics}

The isomorphism relation $u =_\sigma v$ is challenging to define in complete generality.

\vfill
But we know we want the following substitution rule.

\vfill
\centerline{
\unnamed{
\ant{\Sigma \vdash \intype{\sigma}{\type}}
\ant{\Sigma;\;\intype{x}{\tau}\vdash \intype{e[x]}\sigma}
\ant{\Sigma \vdash a=_\tau \;b}
}{
\ant{\Sigma \vdash e[a] =_\sigma e[b]}}
}

\slide{The Foundations of Mathematics}

Independent of the utility of mathematical foundations in the quest for AGI, mathematical truth seems important.

\vfill
The philosophy of the mathematics is difficult.

\vfill
How do we have access to mathematical truth?

\vfill
Are mathematical theorems really true?

\vfill
Is our access to mathematical truth based on innate inference principles --- a particular architecture of thought?


\slide{END}

}
\end{document}
