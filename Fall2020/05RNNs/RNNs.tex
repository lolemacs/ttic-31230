\input ../../SlidePreamble
\input ../../preamble

\begin{document}

{\Huge

  \centerline{\bf TTIC 31230, Fundamentals of Deep Learning}
  \bigskip
  \centerline{David McAllester, Winter 2020}
  \vfill
  \centerline{\bf Recurrent Neural Networks (RNNs)}

\slide{Word Embeddings}

Each word $w$ is associated with a vector $e[w,I]$ called the embedding of word $w$.

\vfill
The matrix $e$ can be viewed as a dictionary assigning each word $w$ the vector $e[w,I]$.

\slide{Recurrent Neural Network (RNN) Language Modeling}

\centerline{\includegraphics[width=3.5in]{\images/RNN}}
\centerline{{\large [Christopher Olah]}}

\vfill
A typical neural language model has the form

$$P_\Phi(w_t\;|\;w_1,\cdots,w_{t-1}) = \softmax_{w_t} e[w_t,I]h[t-1,I]$$

\slide{Vanilla RNNs}

\centerline{\includegraphics[width=3.5in]{\images/RNN}}
\centerline{{\large [Christopher Olah]}}

A Vanilla RNN uses two-input linear threshold units.

{\huge
\begin{eqnarray*}
{\color{red} h[b,t,j]}
& = & \sigma\left(W^{h,h}[j,I]{\color{red} h[b,t-1,I]} + W^{x,h}[j,K]{\color{red} x[b,t,K]} - B[j]\right) \\
\end{eqnarray*}
}

\slideplain{Exploding and Vanishing Gradients}

\vfill
If we avoid saturation of the activation functions then we get exponentially growing or shrinking eigenvectors of the weight matrix.

\vfill
Note that if the forward values are bounded by sigmoids or tanh then they cannot explode.

\vfill
However the gradients can still explode.

\slide{Exploding Gradients: Gradient Clipping}

\vfill
We can dampen the effect of exploding gradients by clipping them before applying SGD.

\vfill
$$W.\mathrm{grad'} = \left\{\begin{array}{l} W.\mathrm{grad} \;\;\;\mbox{if $||W.\mathrm{grad}|| \leq n_{\mathrm{max}}$} \\
                                                      \\ \\
                                                      n_{\mathrm{max}} \; W.\mathrm{grad} / ||W.\mathrm{grad}|| \;\; \mbox{otherwise}
\end{array} \right.$$

\vfill
See {\tt torch.nn.utils.clip\_grad\_norm}

\slide{Time as Depth}

\centerline{\includegraphics[width=3.5in]{\images/RNN}}
\centerline{{\large [Christopher Olah]}}

\vfill
We would like the RNN to {\color{red} remember and use} information from much earlier inputs.


\vfill
All the issues with depth now occur through time.

\vfill
However, for RNNs {\color{red} at each time step we use the same model parameters.}

\vfill
In CNNs {\color{red} at each layer uses its own model parameters.}

\slide{``Residual Connections'' Through Time}

\centerline{\includegraphics[width=3.5in]{\images/RNN}}
\centerline{{\large [Christopher Olah]}}

\vfill
We would like to have residual connections through time.

\vfill
However, we have to handle the fact that the same model parameters are used at every time step.


\slideplain{Gated RNNs}

\centerline{\includegraphics[width=3.5in]{\images/RNN}}
\centerline{{\large [Christopher Olah]}}

$$h[b,t,j] =  {\color{red} G_t[b,t,j]}h[b,t\!-\!1,j] + {\color{red} (1-G[b,t,j])}R[b,t,j]$$

\vfill
This is analogous to a residual connection.

\vfill
Rather than add the ``next layer'' $R[b,t,j]$ to the input $h[b,t\!-\!1,j]$ as in a residual connection, we take a convex combination determined by a computed ``gate'' $G[b,t,j] \in [0,1]$.

\slideplain{Update Gate RNN (UGRNN)}

{\huge
\begin{eqnarray*}
R[b,t,j] & = & \mathrm{tanh}\left(W^{h,R}[j,I]{\color{red} h[b,t\!-\!1,I]} + W^{x,R}[j,K]{\color{red} x[b,t,K]} - B^R[j]\right) \\
\\
G[b,t,j] & = & \sigma\left(W^{h,G}[j,I]{\color{red} h[b,t\!-\!1,I]} + W^{x,G}[j,K]{\color{red} x[b,t,K]} - B^R[j]\right) \\
\\
h[b,t,j] & = & {\color{red} G[b,t,j]}h[b,t\!-\!1,j] + {\color{red} (1-G[b,t,j])}R[b,t,j] \\
\\
{\color{red} \Phi} & {\color{red} =} & {\color{red} (W^{h,R},W^{x,R},B^R,W^{h,G},W^{x,G},B^G)}
\end{eqnarray*}
}
$$\mathrm{tanh}(x) \in (-1,1)\;\;\;\;\;\sigma(x) \in (0,1)$$

\slide{Hadamard product}

$$h[b,t,j] =  {\color{red} G[b,t,j]}h[b,t\!-\!1,j] + {\color{red} (1-G[b,t,j])}R[b,t,j]$$

\vfill
is sometimes written as

$$h[b,t,J] =  G[b,t,J]{\color{red} \odot} h[b,t\!-\!1,J] + (1-G[b,t,J]){\color{red} \odot} R[b,t,J]$$

\vfill
$\odot$ is the Hadamard product (componentwise product) on vectors.

\slide{Gated Recurrent Unity (GRU) by Cho et al. 2014}

\centerline{\includegraphics[width=4.0in]{\images/GRU}}
\centerline{{\huge [Christopher Olah]}}

\vfill
The right half is a UGRNN.

\vfill
The GRU adds a gating on $h_{t-1}$ before the tanh.

\slide{Long Short Term Memory (LSTM)}
\centerline{\includegraphics[width=3.5in]{\images/LSTM}}
\centerline{{\large [figure: Christopher Olah]}}

\centerline{\Large [LSTM: Hochreiter\&Shmidhuber, 1997]}

\slide{UGRNN vs. GRUs vs. LSTMs}

\vfill
In class projects from previous years, GRUs consistently outperformed LSTMs.

\vfill
A systematic study [Collins, Dickstein and Sussulo 2016] states:

\begin{quotation}
  Our results point to the GRU as being the most learnable of gated RNNs for shallow architectures, followed by the UGRNN.
\end{quotation}

\slide{RNN for a generic CELL Procedure}
As usual, we use capital letter indeces to denote whole tensors or slices and lower case letters to denote particular index values.
\vfill
Procedure $\mathrm{RNN}_\Phi(x(T,I))$

\begin{eqnarray*}
h[0,J] &  = &  \mathrm{CELL}_{\Phi.\mathrm{cell}}(\Phi.\mathrm{init}[J],\;x[0,I]) \\
\mathrm{for}\;t>0\;\;h[t,J] &  =  & \mathrm{CELL}_{\Phi.\mathrm{cell}}(h[t-1,J],\;x[t,I])
\end{eqnarray*}

\vfill
Return $h[T,J]$



\slide{bidirectional RNNS}

\centerline{\includegraphics[width = 3in]{\images/biRNN}}

\begin{eqnarray*}
\vec{h}[T,J] & = & \vec{\mathrm{RNN}}_{\Phi.LR}(x[T,I]) \\
\\
\cev{h}[T,J] & = & \cev{\mathrm{RNN}}_{\Phi.RL}(x[T,I]) \\
\\
h[t,2J] & = & \vec{h}[t,J];\cev{h}[t,J]
\end{eqnarray*}

\slide{Multi-Layer RNNs}

\centerline{\includegraphics[width = 1.75in]{\images/RNNstack}}

We let $\ell$ indicate the layer of a multi-layer RNN.
\begin{eqnarray*}
h[0,T,J] & = & \mathrm{RNN}_{\Phi[0]}(x[T,I]) \\
\mbox{for} \;\ell > 0\;\;h[\ell,T,J] & = & \mathrm{RNN}_{\Phi[\ell]}(h[\ell-1,T,J])
\end{eqnarray*}

\vfill
Each layer can be bidirectional.

\slide{Residual Multi-Layer RNNs}

But layers are now typically stacked using residual connections.

\vfill
\begin{eqnarray*}
h[0,T,J] & = & \mathrm{RNN}_{\Phi[0]}(x[T,I]) \\
\\
\mbox{for} \;\ell > 0\;\;h[\ell,T,J] & = & h[\ell-1,T,J] + \mathrm{RNN}_{\Phi[\ell]}(h[\ell-1,T,J])
\end{eqnarray*}

\slide{END}
}
\end{document}
