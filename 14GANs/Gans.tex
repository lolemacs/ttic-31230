\input ../SlidePreamble
\input ../preamble


\begin{document}

{\Huge

  \centerline{\bf TTIC 31230, Fundamentals of Deep Learning}
  \bigskip
  \centerline{David McAllester, Winter 2020}
  \vfill
  \centerline{\bf Generative Adversarial Networks (GANs)}
\vfill
\vfill

\slide{Representing a Distribution with a Generator}
\centerline{$z\sim {\cal N}(0,I)$ \hspace{7em} $y\sim p_\Phi$}
\centerline{\includegraphics[width=6in]{../images/halfdeconv}}


\slide{Generative Adversarial Networks (GANs)}

Let $y$ range over images.  We have a generator $p_\Phi$. For $i \in \{-1,1\}$ we define a probability distribution over pairs
$\tuple{y,i}$ by
\begin{eqnarray*}
\tilde{p}_\Phi(i = 1) & = & 1/2 \\
\tilde{p}_\Phi(y|i=1) & =&  \popd(y) \\
\tilde{p}_\Phi(y|i=-1) & = & p_\Phi(y)
\end{eqnarray*}

\vfill
We also have a discriminator $P_\Psi(i|y)$.
{\color{red} $$\Phi^* = \argmax_\Phi\;\;\min_\Psi\;E_{\tuple{y,i} \sim \tilde{p}_\Phi}\;-\ln P_\Psi(i|y)$$}

\vfill
\centerline{Assuming universality of both $p_\Phi$ and $P_\Psi$ we have {\color{red} $p_{\Phi^*} = \popd$}.}

\slide{The Discrimator Tends to Win}

The log loss for the binary discrimination classifier is quickly driven to very near zero.

\vfill
This causes the learning gradient to also become essentially zero and the learning stops.

\slide{Review of Binary Classification}

In the case of binary classification cross-entropy loss becomes the log loss of the margin

\begin{eqnarray*}
\Psi^* & = & \argmin_\Psi \;\expectsub{(i,y) \sim \tilde{p}_\Phi}{- \ln P_\Psi(i|y)} \\
\\
& = & \argmin_\Psi \expectsub{(i,y) \sim \tilde{p}_\Phi}{\;\;\;\ln(1 + e^{-m}}) \\
\\
m & = & 2is_\Psi(i|y)\;\;\;\mbox{for $\begin{array}{rcl} s_\Psi(-1|y) & = & -s_\Psi(1|y) \\\;P_\Psi(i|y) & = & \softmax_i\;s_\Phi(i|y) \end{array}$}
\end{eqnarray*}


\centerline{\includegraphics[height= 1.5in]{../images/logloss}}

\slideplain{Vanishing Gradients}

For $i = 1$ and $y \sim \popd$:
$$\Psi \;\pluseq \; \eta\;\frac{e^{-m}}{1 + e^{-m}}\;\nabla_\Psi \; m \;\;\approx 0 \;\mbox{for}\;m >> 1$$

\vfill
For $i = -1$ and $y \sim p_\Phi$:

$$\Psi \;\pluseq \; \eta\;\frac{e^{-m}}{1 + e^{-m}}\;\nabla_\Psi\;m \;\;\approx 0 \;\mbox{for}\;m >> 1$$

\vfill
$$\Phi \;\minuseq \; \eta\;\frac{e^{-m}}{1 + e^{-m}}\;\nabla_\Phi\;m \;\;\approx 0 \;\mbox{for}\;m >> 1$$

\vfill
The gradients vanish when the discriminator achieves large margins.

\slide{A Heuristic Patch}

Replace

$$\Phi \;\minuseq \; \eta\;\frac{e^{-m}}{1 + e^{-m}}\;\nabla_\Phi\; m \;\;\approx 0 \;\mbox{for}\;m >> 1$$

\vfill
with

$$\Phi \;\minuseq \; \eta\;\nabla_\Phi\;m$$

\vfill
This allows the generator to recover.

\slidetwo{Generative Adversarial Nets}{Goodfellow et al., June 2014}
\centerline{\includegraphics[width = 9in]{../images/GAN2014}}
The rightmost column (yellow boarders) gives the nearest neighbor in the training data to the adjacent column.

\slide{Assuming Universality of $\Psi$ Only}

\begin{eqnarray*}
\Psi^*(\Phi) & = & \argmin_\Psi  E_{(i,y) \sim \tilde{p}_\Phi}\; - \ln \tilde{p}_\Phi(i|y) \\
\\
\tilde{p}_\Phi(i|y) & = & \frac{\tilde{p}_\Phi(i \wedge y)}{\tilde{p}_\Phi(y)} \\
\\
\\
\tilde{p}(1|y) & = & \frac{\frac{1}{2}\popd(y)}{\frac{1}{2}\popd(y) + \frac{1}{2}p_\Phi(y)} \\
\\
\\
\tilde{p}(-1|y) & = & \frac{\frac{1}{2}p_\Phi(y)}{\frac{1}{2}\popd(y) + \frac{1}{2}p_\Phi(y)}
\end{eqnarray*}

\slide{Assuming Universality of $\Psi$ Only}

\begin{eqnarray*}
& & E_{(i,y) \sim  p_\Phi}\;\ln \tilde{p}_\Phi(i|y) \\
\\
& = & \frac{1}{2} E_{y \sim \popd}\; \ln \frac{\frac{1}{2}\popd(y)}{\frac{1}{2}\popd(y) + \frac{1}{2} p_\Phi(y)} + \frac{1}{2} E_{y \sim p_\Phi}\;\ln \frac{\frac{1}{2}p_\Phi(y)}{\frac{1}{2}\popd(y) + \frac{1}{2}p_\Phi(y)} \\
\\
\\
& = & \frac{1}{2} \left(\;KL\left(\popd,\frac{\popd+p_\Phi}{2}\right),\;KL\left(p_\Phi,\frac{\popd+p_\Phi}{2}\right)\right) - \ln 2 \\
\\
& = & {\color{red} \mathrm{JSD}(\popd,p_\Phi) - \ln 2}
\end{eqnarray*}

\slide{Assuming Universality of $\Psi$ Only}

\begin{eqnarray*}
& &  E_{(i,y) \sim \tilde{p}_\Phi}\; - \ln \tilde{p}_\Phi(i|y) \\
\\
& = & \ln 2 - JSD(\popd,p_\Phi)
\end{eqnarray*}


\slide{Contrastive GANs}

A GAN can be built with a ``contrastive'' discriminator.  Rather than estimate the probability that $y$ is from the population, the discriminator must select which
of $y_1,\ldots,y_N$ is from the population.

\vfill
More formally, for $N \geq 2$ let {\color{red} $\tilde{p}_\Phi^{(N)}$} be the distribution defined by drawing one ``positive'' from $\popd$ and $N-1$ IID negatives from $p_\Phi$;
then inserting the positive at a random position among the negatives; and returning $(i,y_1,\ldots,y_N)$ where
$i$ is the index of the positive.

\anaslide{Contrastive GANs}

\begin{eqnarray*}
\Psi^*(\Phi) & = & \argmin_\Psi\; E_{(i,y_1,\ldots,y_{N})\sim \tilde{p}_\Phi^{(N)}} \;- \ln P_\Psi(i|y_1,\ldots,y_{N}) \\
\\
\Phi^* & = & \argmax_\Phi\; E_{(i,y_1,\ldots,y_{N})\sim \tilde{p}_\Phi^{(N)}} \;- \ln P_{\Psi^*(\Phi)}(i|y_1,\ldots,y_{N})
\end{eqnarray*}

\vfill
{\color{red} $\tilde{p}_\Phi^{(2)}(i|y_1,y_2)$} requires a choice between two $y$'s while {\color{red} $\tilde{p}_\Phi(i|y)$} classifies a single $y$ --- these are different.

\vfill
The discrimination gets more difficult as $N$ gets larger.

\anaslide{Contrastive GANs}

\begin{eqnarray*}
\Psi^*(\Phi) & = & \argmin_\Psi\; E_{(i,y_1,\ldots,y_{N})\sim \tilde{p}_\Phi^{(N)}} \;- \ln P_\Psi(i|y_1,\ldots,y_{N}) \\
\\
\Phi^* & = & \argmax_\Phi\; E_{(i,y_1,\ldots,y_{N})\sim \tilde{p}_\Phi^{(N)}} \;- \ln P_{\Psi^*(\Phi)}(i|y_1,\ldots,y_{N})
\end{eqnarray*}

\vfill
Assuming universality

\vfill
$${\cal L}_{\mathrm{Discr}}(\Psi^*(\Phi)) = H_\Phi(i|y_1,\ldots y_{N})$$

\vfill
$$p_{\Phi^*} = \popd\;\;\;\;\;\;H_{\Phi^*}(i|y_1,\ldots,y_N) = \ln N$$

\slidetwo{Noise Contrastive Estimation}{Gutmann and Hyv\"{a}rinen, 2010}

{\color{red}
$$\Psi^* = \argmin_\Psi E_{(i,y_1,\ldots,y_{N}) \sim \tilde{p}_\Phi^{(N)}} \;- \ln P_\Psi(i|y_1,\ldots,y_{N})$$
\centerline{$p_\Phi$ is fixed ``noise''}
}

\vfill
Assume $p_\Phi$ is both samplable and computable --- we can sample from $p_\Phi$ and for any given $y$ we can compute $p_\Phi(y)$.

\vfill
Assume $P_\Psi(i|y_1,\ldots,y_{N}) = \softmax_i s_\Psi(y_i)$

\vfill
Assume $\Psi$ universal


\anaslide{Noise Contrastive Estimation}

$$\Psi^* = \argmin_\Psi E_{(i,y_1,\ldots,y_{N}) \sim \tilde{p}_\Phi^{(N)}} \;- \ln P_\Psi(i|y_1,\ldots,y_{N})$$
\centerline{$p_\Phi$ is fixed ``noise''}

\vfill
Theorem: {\color{red} $\popd(y) = \softmax_y\;\;\; s_{\Psi^*}(y) + \ln p_\Phi(y)$}

\vfill
We then have a computable score function (energy function) for the population.  We do not have the partition function $Z$.

\anaslide{Noise Contrastive Estimation}

$$\Psi^* = \argmin_\Psi E_{(i,y_1,\ldots,y_{N}) \sim \tilde{p}_\Phi^{(N)}} \;- \ln P_\Psi(i|y_1,\ldots,y_{N})$$
\centerline{$p_\Phi$ is fixed ``noise''}

\vfill
Lemma: {\color{red} $P_{\Psi^*}(i|y_1,\ldots,y_{N})  =  \softmax_i\;\; \ln \frac{\popd(y_i)}{p_\Phi(y_i)}$}

\slide{Lemma Proof}
{\huge
\begin{eqnarray*}
{\color{red} \tilde{p}_\Phi^{(N)}(i \;\mathrm{and}\;y_1,\ldots,y_{N})} & = & \frac{1}{N}\;\popd(y_i)\prod_{j \not = i}p_\Phi(y_j) \\
\\
& = & \alpha \frac{\popd(y_i)}{p_\Phi(y_i)},\;\;\;\alpha = \frac{1}{N} \prod_i p_\Phi(y_i) \\
\\
\\
{\color{red} \tilde{p}_\Phi^{(N)}(i\;|\;y_1,\ldots y_{N})} & = & \frac{\tilde{p}_\Phi^{(N)}(i\;\mbox{and} \; y_1,\ldots,y_{N})}{\sum_i \;\tilde{p}_\Phi^{(N)}(i \;\mbox{and}\; y_1,\ldots,y_{N})} \;=\;\frac{1}{Z} \;\frac{\popd(y_i)}{p_\Phi(y_i)} \\
\\
\\
& = & {\color{red} \softmax_i\; \left(\ln \frac{\popd(y_i)}{p_\Phi(y_i)}\right)}
\end{eqnarray*}
}

\slide{Theorem Proof}

$$\softmax_i s_{\Psi^*}(y_i) =  \softmax_i \ln \frac{\popd(y_i)}{p_\Phi(y_i)}$$

\vfill
is solved by

\vfill
$$s_{\Psi^*}(y) = \ln \frac{\popd(y)}{p_\Phi(y)} - \ln Z$$

\vfill
giving

$$\popd(y) = \frac{1}{Z} \exp(s_\Psi(y) + \ln p_\Phi(y))$$

\slide{Another Theorem}

\begin{eqnarray*}
  && E_{(i,y_1,\dots,y_{N}) \sim \tilde{p}_\Phi^{(N)}}\;- \ln p^{(N)}_{\Psi^*}(i|y_1,\ldots,y_{N}) \\
  \\
  \\
  \\
  & \geq & \ln N - \frac{N-1}{N}(KL(\popd,p_\Phi) + KL(p_\Phi,\popd))
\end{eqnarray*}

\vfill
Note that the bound holds with equality for $p_\Phi = \popd$.

\vfill
This is analogous to the JSD expression for the optimal discriminator loss.
\slide{Proof Part A.}

{\huge
 \begin{eqnarray*}
    & & E_{(i,y_1,\ldots,y_{N}) \sim \tilde{p}_\Phi^{(N)}}\;\ln p_{\Psi^*}(i|y_1,\ldots,y_{N}) \\
    \\
    & = & E_{(i,y_1,\ldots,y_{N}) \sim \tilde{p}_\Phi^{(N)}}\;\ln \left(\softmax_i \;\ln \frac{\popd(y_i)}{p_\Phi(y_i)}\right)[i] \\
    \\
    & = & E_{(i,y_1,\ldots,y_{N}) \sim \tilde{p}_\Phi^{(N)}}\;\ln\frac{\popd(y_i)}{p_\Phi(y_i)} - \ln\left(\sum_j \frac{\popd(y_j)}{p_\Phi(y_j)} \right) \\
    \\
    & = & E_{(i,y_1,\ldots,y_{N}) \sim \tilde{p}_\Phi^{(N)}}\;\ln\frac{\popd(y_i)}{p_\Phi(y_i)} - \ln\left(\frac{1}{N}\sum_j \frac{\popd(y_j)}{p_\Phi(y_j)} \right) - \ln\;N
  \end{eqnarray*}
}

\slide{Proof Part B.}
{\huge
 \begin{eqnarray*}
    & & E_{(i,y_1,\ldots,y_{N}) \sim \tilde{p}_\Phi^{(N)}}\;\ln\frac{\popd(y_i)}{p_\Phi(y_i)} {\color{red} - \ln\left(\frac{1}{N}\sum_j \frac{\popd(y_j)}{p_\Phi(y_j)} \right)} - \ln\;N \\
    \\
    & {\color{red} \leq} & E_{(i,y_1,\ldots,y_{N}) \sim \tilde{p}_\Phi^{(N)}}\;\ln\frac{\popd(y_i)}{p_\Phi(y_i)} {\color{red} - \frac{1}{N} \sum_j \ln\frac{\popd(y_j)}{p_\Phi(y_j)}} - \ln\;N \\
    \\
    & = & E_{y \sim \popd} \ln \frac{\popd(y)}{p_\Phi(y)} -  E_{(i,y_1,\ldots,y_N) \sim \tilde{p}^{(N)}_\Phi} \frac{1}{N} \sum_j \ln\frac{\popd(y_j)}{p_\Phi(y_j)} - \ln N \\
    \\
    & = & \frac{N-1}{N}(KL(\popd,p_\Phi) + KL(p_\Phi,\popd)) - \ln\;N \\
  \end{eqnarray*}
}

\slidetwo{Unsupervised Representation Learning ... (DC GANS)}
{Radford et al., Nov. 2015}

\centerline{\includegraphics[width = 9in]{../images/GANDCa}}

\slidetwo{Unsupervised Representation Learning ... (DC GANS)}
{Radford et al., Nov. 2015}

\centerline{\includegraphics[width = 9in]{../images/ImageFeatures}}

\slide{Interpolated Faces}

[Ayan Chakrabarti, January 2017]

\centerline{\includegraphics[height = 4.5in]{../images/interp}}

\slidetwo{Progressive Growing of GANs}{Karras et al., Oct. 2017}
\centerline{\includegraphics[height = 4.5in]{../images/GANproga}}

\slidetwo{Progressive Growing of GANs}{Karras et al., Oct. 2017}
\centerline{\includegraphics[height = 4.5in]{../images/GANprogb}}

\slidetwo{Progressive Growing of GANs}{Karras et al., Oct. 2017}
\centerline{\includegraphics[height = 4.5in]{../images/GANprogc}}


\slide{Early GANs on ImageNet}

\centerline{\includegraphics[height = 4.5in]{../images/BadGAN}}


\slidetwo{Large Scale GAN Training}{Brock et al., Sept. 2018}
\centerline{\includegraphics[width = 9in]{../images/GANclass}}

\vfill
This is a class-conditional GAN --- it is conditioned on the imagenet class label.

\vfill
This generates 512 X 512 images without using progressive training.
\slide{Issues}

\centerline{Mode Collapse}

\vfill
\centerline{Unstable Training}

\vfill
\centerline{Measuring Perfomance}

\ignore{

\slidetwo{Converting to Cross Entropy}{Goodfellow, 2014}

In Goodfellow's original paper he expressed a preference for cross entropy loss (the fundamental equation) over Jensen-Shannon loss.

$${\color{red} \Phi^* = \argmin_\Phi\;\mathrm{JSD}(\pop,p_\Phi)}$$

\centerline{vs.}
\vspace{-2ex}
$${\color{red} \Phi^* = \argmin_\Phi\;H(\pop,p_\Phi)}$$

{\color{red}
$$\begin{array}{lrcl}
\mathrm{GAN:} & \Phi^* & = & \argmin_\Phi\;\mathrm{JSD}(p_\Phi,\pop) \\
\\
\mathrm{GAN':} & \Phi^* & = & \argmin_\Phi H(\pop, p_\Phi) \\
\\
\mathrm{contrastiveGAN:} & \Phi^* & = & \argmin_\Phi KL(p_\Phi,\pop)
\end{array}$$
}

\vfill
He presented a modification to the GAN adversarial objective that yields cross-entropy loss rather than Jensen-Shanon loss.

\slidetwo{Converting to Cross Entropy}{Goodfellow, 2014}

{\color{red} $$\Psi^*(\Phi) = \argmin_\Psi \;\expectsub{(i,y) \sim \tilde{p}_\Phi}{- \ln P_\Psi(i|y)}$$}

\vfill
$$\mbox{Assume:}\;\;\; P_{\Psi^*}(1|y) = \frac{\mathrm{Pop}(y)}{\mathrm{Pop}(y) + p_\Phi(y)}$$

\vfill
\begin{eqnarray*}
  \mbox{Define:}\;\;\; {\color{red} f_{\Psi^*}(y)} & \doteq & \frac{p_{\Psi^*}(1|y)}{p_{\Psi^*}(-1|y)} \\
\\
& = & {\color{red} \frac{\mathrm{Pop}(y)}{P_\Phi(y)}}
\end{eqnarray*}

\slide{Converting to Cross Entropy}

{\huge
\begin{eqnarray*}
 {\color{red} \Phi^*}& = & {\color{red} \argmax_\Phi\;E_{y \sim \pop} \;f_{\Psi^*}(y)} \\
 \\
 {\color{red} \nabla_\Phi \; E_{y \sim p_\Phi}\;  f_{\Psi^*}(y)}  & = & \nabla _\Phi \sum_y\; p_\Phi(y) f_{\Psi^*}(y) \\
  \\
  & = & \sum_y \; p_\Phi(y) f_{\Psi^*}(y) \nabla_\Phi \ln p_\Phi(y) \\
  \\
  & = & \sum_y \;\mathrm{Pop}(y) \nabla_\Phi \ln p_\Phi(y) \\
  \\
  & = & E_{y \sim \mathrm{Pop}} \; \nabla_\Phi \ln p_\Phi(y) \\
  \\
  & = & {\color{red} \nabla_\Phi \; E_{y \sim \mathrm{Pop}} \;\ln p_\Phi(y)}
\end{eqnarray*}
}
}


\slide{Mode Collapse a.k.a Mode Dropping}

The generator distribution drops portions of the population.

\centerline{\includegraphics[width=9in]{../images/Unstable1}}

\slide{Unstable Training}

Joint SGD is not the same as nested max-min.

\vfill
Consider
$${\color{red} \max_x \; \min_y\; xy}$$

\vfill
A Nash equilibrium is $x= y = 0$.

\vfill
Simultaneous gradient flow yields

{\color{red} $$\frac{dx}{dt}  =  y \;\;\;\;\;\;\frac{dy}{dt} = -x$$}

\vfill
This goes in a circle.

\slide{Unstable Training}

The generator distribution drifts as the discriminator follows.

\centerline{\includegraphics[width=9in]{../images/Unstable1}}

\slidetwo{Pros and Cons of GAN Evaluation Measures}{Borji, Oct 2018}

We would like a rate-distortion metric on distribution models.

\vfill
This has not yet been achieved for GANs.

\vfill
Evaluation of GANs always involves, at least in part, subjective judgments of naturalness.

\vfill
Sometimes automated metrics are also used.

\vfill
The above paper discusses various proposed automated metrics of GAN performance.  Current automated metrics are questionable.


\slidetwo{Image-to-Image Translation (Pix2Pix)}
{Isola et al., Nov. 2016}

We assume a corpus of ``image translation pairs'' such as images paired with semantic segmentations.

\centerline{\includegraphics[width = 8.0in]{../images/cGAN0}}

\slide{Conditional GANS}
All unconditional distribution modeling methods apply to conditional distribution modeling.
\vfill

Let $y$ range over images.  We have a generator $p_\Phi$. For $i \in \{-1,1\}$ we define a probability distribution over triple
$\tuple{x,y,i}$ by
\begin{eqnarray*}
\tilde{p}_\Phi(i = 1) & = & 1/2 \\
\tilde{p}_\Phi(y|i=1) & =&  \popd(y|x) \\
\tilde{p}_\Phi(y|i=-1) & = & p_\Phi(y|x)
\end{eqnarray*}

\vfill
We also have a discriminator $P_\Psi(i|y)$.
{\color{red} $$\Phi^* = \argmax_\Phi\;\;\min_\Psi\;E_{\tuple{x,y,i} \sim \tilde{p}_\Phi}\;-\ln P_\Psi(i|x,y)$$}

\slide{Conditional GANs}

$${\color{red} \Phi^* = \argmin_\Phi \; \max_\Psi \;\expectsub{x,y,i \sim \tilde{p}_\Phi(y|x)}\;
  {\ln P_\Psi(i|x,y)}}$$

\slide{Adversarial Discrimination as an Additional Loss}

$${\color{red} \Phi^* = \argmin_\Phi\;E_{(x,y) \sim \popd}\;\; ||x - y||^2\; + \; \lambda\; {\cal L}_{\mathrm{Discr}}(\Phi)}$$

\vfill
$${\cal L}_\mathrm{Discr}(\Phi) = \max_\Psi \;E_{x,y,i \sim \tilde{p}_\Phi}\; \ln P_\Psi(i|y,x)$$

\slide{Discrimination as an Additional Loss}

{\huge
$$\begin{array}{lrcl}
\mathrm{L1:} & \Phi^* & = & \argmin_\Phi\;E_{(x,y) \sim \popd}\;\; ||y - y_\Phi(x)||_1 \\
\\
\\
\mathrm{cGAN:} & \Phi^* & = & \argmin_\Phi\;{\cal L}_{\mathrm{Discr}}(\Phi) \\
\\
\\
\mathrm{L1 + cGAN:} & \Phi^* & = & \argmin_\Phi\;E_{(x,y) \sim \popd}\;\; ||y - y_\Phi(x)||_1\; +\; \lambda\; {\cal L}_\mathrm{Discr}(\Phi)
\end{array}$$
}


\slidetwo{Image-to-Image Translation (Pix2Pix)}
{Isola et al., Nov. 2016}

\centerline{\includegraphics[height = 4.5in]{../images/cGAN1}}

\slide{Arial Photo to Map and Back}

\centerline{\includegraphics[width = 8.0in]{../images/cGAN2}}

\slide{Semantic Segmentation}

\centerline{\includegraphics[width = 8.0in]{../images/cGAN4}}

\ignore{
\slidetwo{Unpaired Image-to-Image Translation (Cycle GANs)}{Zhu et al., March 2017}

We have two corpora of images, say images of zebras and unrelated images of horses, or photographs and unrelated paintings by Monet.

\vfill
We want to construct translations between the two classes.

\centerline{\includegraphics[width = 8.0in]{../images/Cycle2}}

\slide{Cycle Gans}

\centerline{\includegraphics[width = 11.0in]{../images/Cycle3}}

\slide{Cycle Gans}

\centerline{\includegraphics[width = 6.0in]{../images/Cycle4}}

\slidetwo{Unsupervised Machine Translation (UMT)}
         {Lample et al, Oct. 2017, also Artetxe et al., Oct. 2017}

\centerline{\includegraphics[width = 10.0in]{../images/Cycle5}}
}

\slidetwo{Feature Alignment by Discrimination}{Text to Speech (Saito et al. Sept. 2017)}

\centerline{\includegraphics[width = 2.0in]{../images/Txt2spchGAN}}

\vfill
Minimum Generation Error (MGE) uses {\color{red} perceptual distortion} ---
a distance between the feature vector of the generated sound wave and the
feature vector of the original.

\vfill
{\color{red}Perceptual Naturalness} can be enforced by a feature discrimination loss.

\slidetwo{Adversarial Discriminative Domain Adaptation}{Tzeng et al. Feb. 2017}

\centerline{\includegraphics[width = 4.0in]{../images/AdvDomainAdapt}}

A feature discrimination loss can be used to align source and target features.

\slide{Comments}

\vfill
I predict that in a few years adversarial discrimination will be limited to enforcing perceptual naturalness in the generation of sounds and images.

\vfill
Cooperative discrimination seems more useful for predictive tasks. We will see that cooperative discrimination has been effective in pretraining.
\slide{END}

}
\end{document}
