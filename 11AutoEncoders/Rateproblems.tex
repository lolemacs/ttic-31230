\documentclass{article}
\input ../preamble
\parindent = 0em
\parskip = 1ex

%\newcommand{\solution}[1]{}
\newcommand{\solution}[1]{\bigskip {\color{red} {\bf Solution}: #1}}

\begin{document}


\centerline{\bf TTIC 31230 Fundamentals of Deep Learning}

\bigskip

\centerline{\bf Problems for RDAs and VAEs}

\bigskip
\bigskip

{\bf Problem 1.  Mutual Information as Channel Capacity}

The mutual information between two random variables $x$ and $y$ is defined by
$$I(x,y) = E_{x,y}\;\ln\frac{P(x,y)}{P(x)P(y)} = KL(P(x,y),P(x)P(y))$$
Mutual information has an interpretation as a channel capacity.

\medskip
Suppose that
we draw a random bit $y \in \{0,1\}$ with $P(0) = P(1) = 1/2$ and send it across a noisy channel
to a receiver who gets $y' = y \oplus \epsilon$ where $\epsilon$ is an independent ``noise variable'' with $\epsilon \in\{0,1\}$, where
$\oplus$ is exclusive or ($y$ gets flipped when $\epsilon = 1$),
and where the ``noise'' $\epsilon$ has a probability $P$ of being 1.

\medskip(a) Solve for the channel capacity $I(y,y')$ as a function of $P$ in units of bits.
When measured in bits, this channel capacity has units of bits received per message sent.

\solution{
  \begin{eqnarray*}
    I(y,y') & = & H(y) - H(y|y') \\
    H(y) & = & 1\;\mathrm{bit} \\
    \\
    H(y|y') & = & P(y=y') (- \log_2 P(y=y')) + P(y \not = y') (-log_2 P(y \not = y')) \\
    & = & P(\epsilon = 0) (-\log_2 P(\epsilon = 0)) + P(\epsilon = 1) -log_2 P(\epsilon = 1) \\
    & = & (1-P) \log_2 1/(1-P) + P \log_2 1/P \\
    & = & H(P)
  \end{eqnarray*}
}

\medskip
(b) Explain why your answer to part (a) makes sense in terms of what the receiver knows for $P = 1/2$ and when $P=1$.

\solution{
  For $P = 1/2$ we have $H(P) = 1$ bit and $I(y,y') = H(y) - H(P) = 0$ and the receiver knows nothing about $y$.
  For $P = 1$ we have $H(P) = 0$  and $I(y',y) = 1$ bit.  Note that in this case $y'$ is $1-y$ so $y'$ carries full information about $y$.
}

\vfill
\eject
~{\bf Problem 2. Rate-Distortion Autoencoders}

(a)  Consider an arbitrary distribution $P(z,y)$. Show the variational equation
$$I(y,z) = \inf_q\;E_{y\sim \popd} KL(P_\Phi(z|y),Q(z))$$
where $Q$ ranges over distributions on $z$.
Hint: It suffices to show $$I(y,z) \leq \;E_y KL(P_\Phi(z|y),Q(z))$$
and that there exists a $Q$ achieving equality.

\solution{

  \begin{eqnarray*}
 & & I(y,z) \\
 \\
 & = & E_{y \sim \popd}\; KL(P(z|y),P(z)) \\
\\
& = & E_{y,z\sim P(z|y)}\; \left(\ln \frac{P(z|y)}{{\color{red} Q(z)}} + \ln \frac{{\color{red} Q(z)}}{P(z)}\right) \\
\\
& = & E_{y \sim \popd}\;KL(P(z|y),Q(z)) + \left(E_{y \sim \popd,\;z\sim P(z|y)}\;\ln \frac{Q(z)}{P(z)}\right)
\\
& = & E_y\;KL(P(z|y),Q(z)) + E_{\color{red} z\sim P(z)}\;\ln \frac{Q(z)}{P(z)} \\
\\
& = & E_y\;KL(P(z|y),Q(z)) - KL(P(z),Q(z)) \\
\\
& \leq & E_{y \sim \popd}\; KL(P(z|y),Q(z))
\end{eqnarray*}

Equality is achieved when $Q(z) = P(z)$.
}


\medskip
(b) Consider a rate-distortion autoencoder.
$$\Phi^* = \argmin_\Phi\;I_\Phi(y,z) + \lambda E_{y \sim \popd,\;z \sim P_\Phi(z|y)}\;\mathrm{Dist}(y,y_\Phi(z)).$$
Here $I_\Phi(y,z)$ is defined by the distribution where we draw $y$ from $\popd$ and $z$ from $P_\Phi(z|y)$.
We will write $P_\popd(z)$ for the marginal on $z$ under this distribution.
$$P_\popd(z) = E_{y \sim \pop} \;P_\Phi(z|y)$$

\medskip
Based on the result from part (b) rewrite the above definition of rate-distortion autoencoder to be a minimization over three independent models
$P_\Phi(z)$ and $P_\Phi(y|z)$ and $P_\Phi(z|y)$ (although these models share parameters we will assume that $\Phi$ is sufficiently rich that
the models are independently optimizable).

\solution{
  $$\Phi^* = \argmin_{\Phi} E_{y \sim \popd, z\sim P_\Phi(z|y)}\; \ln \frac{P_\Phi(z|y)}{P_\Phi(z)} + \lambda\;\mathrm{Dist}(y,y_\Phi(z)).$$
}

\bigskip
~{\bf Problem 3. Modeling Rounding with Continuous Noise.}
    
Consider a rate-distortion autoencoder with $y$ and $z$ continuous.
$$\Phi^* = \argmin_{\Phi,\Phi}\;E_{y \sim \pop} KL(p_\Phi(z|y),p_\Phi(z)) + \lambda E_{y \sim \pop,\;z \sim P(z|y)}\;\mathrm{Dist}(y,y_\Phi(z)).$$
Define $p_\Phi(z|y)$ by $z = z_\Phi(y) + \epsilon$ with $z_\Phi[y] \in \mathbb{R}^d$
and $\epsilon$ drawn uniformly from $[0,1]^d$. In other words,
we add noise drawn uniformly from $[0,1]$ to each component of $z_\Phi(y)$.

\medskip
Define $p_\Phi(z)$ to be log-uniform in each dimension.  More specifically
$p_\Phi(z)$ is defined by drawing $s[i]$ uniformly from the interval
$[0,s_{\mathrm{max}}]$ and then setting $z[i] = e^s$ so that $\ln z[i]$ is uniformly distributed over the interval $[0,s_{\mathrm{max}}]$.
This gives
\begin{eqnarray*}
  dz & = & e^sds  \;\;= \;zds\\
  \\
  dp & = & \frac{1}{s_{\mathrm{max}}}\;ds \\
  \\
  p_\Phi(z[i]) & = & \frac{dp}{dz} \;\;= \frac{1}{s_{\mathrm{max}}z[i]}
\end{eqnarray*}

\medskip
Assume That we have that $z_\Phi(y) \in [1,e^{s_{\mathrm{max}}}-1]^d$ so that with probability 1 over the draw of $\epsilon$ we have
$\ln(z_\Phi(y) + \epsilon) \in [0,s_{\mathrm{max}}]$.

\medskip
(a) For $z \in [z_\Phi(y),z_\Phi(y)+1]$ what is $p_\Phi(z|y)$?

\solution{1}

\medskip
(b) Solve for $KL(p_\Phi(z|y),p_\Phi(z))$ in terms of $z_\Phi(y)$ under the above specifications and simplify your answer
for the case of $z_\Phi(y)[i] >> 1$.

\solution{
\begin{eqnarray*}
  & & KL(p_\Phi(z|y),p_\Phi(z)) \\
  \\
  & = & E_{z \sim P_\Phi(z|y)}\;\ln\frac{p_\Phi(z_\Phi(y))}{p_\Phi(z)} \\
  \\
  & = & E_{z \sim P_\Phi(z|y)} \;\sum_i \ln \frac{1}{1/(s_{\mathrm{max}}z[i])} \\
  \\
  & = & \sum_i E_{z[i]}\;\ln (s_{\mathrm{max}}z[i]) \\
  \\
  & = & \left(\sum_i \int_{z_\Phi(y)[i]}^{z_\Phi(y)[i]+1} \ln z\; dz\right) + d \ln s_{\mathrm{max}} \\
  \\
  & = & \left(\sum_i [z\ln z - z]_{z_\Phi(y)[i]}^{z_\Phi(y)[i]+1} \right) + d \ln s_{\mathrm{max}}
  \\
  & = & \left(\sum_i [z\ln z]_{z_\Phi(y)[i]}^{z_\Phi(y)[i]+1} \right) + d \ln s_{\mathrm{max}} - 1 \\
  \\
  & = & \left(\sum_i \ln (z_\Phi(y)[i] + 1) + z_\Phi(y)[i](\ln (z_\Phi(y)[i] + 1) - \ln z_\Phi(y)[i]) \right) + d \ln s_{\mathrm{max}} - 1 \\
  \\
  & = & \left(\sum_i \ln (z_\Phi(y)[i] + 1) + z_\Phi(y)[i]\ln \left(1+ \frac{1}{z_\Phi(y)[i]}\right) \right) + d \ln s_{\mathrm{max}} - 1 \\
  \\
  & \approx & \left(\sum_i \ln z_\Phi(y)[i] \right) + d \ln s_{\mathrm{max}} \;\;\;\mbox{for $z_\Phi(y)[i] >> 1$}
\end{eqnarray*}
}

\end{document}
